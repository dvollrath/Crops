\documentclass[11pt]{article}
%%%%%%%%%%%%%%%%%%%%%%%%%%%%%%%%%%%%%%%%
\usepackage{amsmath}
\usepackage{verbatim}
\usepackage[usenames,dvipsnames]{color}
\usepackage{setspace}
\usepackage{lscape}
\usepackage{longtable}
\usepackage[top=1.25in,bottom=1.25in,left=1in,right=1in]{geometry}
\usepackage{graphicx}
\usepackage{epstopdf}
\usepackage[usenames,dvipsnames]{pstricks}
\usepackage{epsfig}
\usepackage{pstricks-add}
\usepackage{pst-node}
\usepackage{pst-plot}
\usepackage{fancyhdr}
\usepackage[absolute,showboxes]{textpos}
\usepackage{booktabs}
\usepackage{dcolumn}
\usepackage{arydshln}
\usepackage{natbib}
\usepackage{tabularx}

\setcounter{MaxMatrixCols}{10}
\newcolumntype{d}[1]{D{.}{.}{-2.#1}}
\newenvironment{proof}[1][Proof]{\noindent\textbf{#1.} }{\ \rule{0.5em}{0.5em}}
\setlength{\columnsep}{.2in}
\psset{unit=1cm}

\def\sym#1{\ifmmode^{#1}\else\(^{#1}\)\fi}

\begin{document}
\begin{titlepage}
\vspace{2in} \noindent {\large \today}

\vspace{.5in} \noindent {\Large \textbf{\strut The Influence of Crop Type on Comparative Development}}

\vspace{.25in} \noindent {\large T. Ryan Johnson}

\vspace{.05in} \noindent University of Houston

\vspace{.25in} \noindent {\large Dietrich Vollrath}

\vspace{.05in} \noindent University of Houston

\vfill \noindent \textsc{Abstract} \hrulefill

\vspace{.05in} \noindent Stuff

\vspace{.1in} \hrule

\vspace{.5in} \noindent {\small JEL Codes: TBD}

\vspace{.1in} \noindent {\small Keywords: TDB}

\vspace{.1in} \noindent {\small Contact information: 201C McElhinney Hall, U. of Houston, Houston, TX 77204, devollrath@uh.edu.}
\end{titlepage}

\pagebreak 

\clearpage

\begin{table}[!htb]
\begin{center}
\caption{Estimates of Malthusian Tightness, $\beta$, by Region, 2000CE}
\label{TAB_beta_region}
{\small
\begin{tabularx}{\textwidth}{lXXXXXX}
\midrule
\multicolumn{7}{l}{Dependent Variable: Log caloric yield ($A_{ic}$)} \\
 & \multicolumn{6}{c}{Region:} \\ \cmidrule{2-7}
 &        &      & Sub-        & South \&  & North     &  \\
 &        &      & Saharan     & Central   & Africa \& & North \\
 & Europe & Asia & Africa      & America   & Mideast   & America \\
 & (1) & (2) & (3) & (4) & (5) & (6) \\
\midrule
Log rural density   &       0.443&       0.213&       0.146&       0.348&       0.096&       0.394\\
                    &     (0.114)&     (0.070)&     (0.061)&     (0.051)&     (0.061)&     (0.015)\\
\midrule
p-value $\beta=\beta^{Eur}$&           .&       0.087&       0.023&       0.446&       0.009&       0.667\\
Countries           &          34&          23&          44&          18&          26&           2\\
Observations        &         529&         636&         574&         296&         418&          62\\
Adjusted R-square   &        0.53&        0.65&        0.54&        0.65&        0.40&        0.71\\

\midrule
\end{tabularx}
}
\end{center}
\vspace{-.5cm}\singlespacing {\footnotesize \textbf{Notes}: Standard errors are clustered at the country level, *** indicates significance at 1\%, ** at 5\%, and * at 10\%. All regressions include country fixed effects and a constant. See appendix for lists of exact countries included in each region. The coefficient estimate on rural population density indicates the value of $\beta$, see equation (\ref{EQ_regress}). Rural population is from HYDE database \citep{hyde31}, and caloric yield is the author's calculations based on the data from \citet{galorozak2016}. The p-value is from a hypothesis test that the estimated $\beta$ is equal to that estimated for Europe, $\beta^{Eur}$, and is obtained from an interaction term in a separate regression including both Europe and the given region, see equation (\ref{EQ_interaction}) and the text for details.
}
\end{table}


\clearpage
\begin{table}[!htb]
\begin{center}
\caption{Estimates of Malthusian Tightness, $\beta$, by Sub-regions, 2000CE}
\label{TAB_beta_subregion}
{\small
\begin{tabularx}{\textwidth}{lXXXXX}
\midrule
\multicolumn{6}{l}{Dependent Variable in both panels: Log caloric yield ($A_{ic}$)} \\ \\
\\
Panel A & \multicolumn{5}{c}{Sub-Region:} \\ \cmidrule{2-6}
 &          &         &             &  \multicolumn{2}{c}{Excl. China} \\ \cmidrule(lr){5-6}
 & North \& &         &              & South \&  & Central \&             \\
 & Western  & Eastern & Southern     & Southeast & West        \\
 & Europe   & Europe  & Europe       & Asia      & Asia      \\
 & (1) & (2) & (3) & (4) & (5) \\
\midrule
Log rural density   &       0.562&       0.478&       0.119&       0.052&       0.353\\
                    &     (0.185)&     (0.172)&     (0.069)&     (0.012)&     (0.095)\\
\midrule
p-value $\beta=\beta^{NWEur}$&           .&       0.734&       0.031&       0.009&       0.313\\
Countries           &          16&           9&           9&          10&          11\\
Observations        &         177&         216&         136&         343&         201\\
Adjusted R-square   &        0.64&        0.32&        0.29&        0.40&        0.59\\

\midrule
\\
Panel B & \multicolumn{5}{c}{Sub-Region:} \\ \cmidrule{2-6}
 &           &   &           &          &             \\
 & Temperate & Tropical  & Tropical & South    & North    \\
 & Americas  & Americas  & Africa   & Africa   & Africa     \\
\midrule
Log rural density   &       0.394&       0.096&       0.133&       0.329&       0.385\\
                    &     (0.015)&     (0.061)&     (0.064)&     (0.068)&     (0.074)\\
\midrule
p-value $\beta=\beta^{NWEur}$&       0.371&       0.018&       0.028&       0.245&       0.374\\
Countries           &           2&          26&          40&           4&           5\\
Observations        &          62&         418&         539&          35&          94\\
Adjusted R-square   &        0.71&        0.40&        0.56&        0.22&        0.53\\

\midrule
\end{tabularx}
}
\end{center}
\vspace{-.5cm}\singlespacing {\footnotesize \textbf{Notes}: Standard errors are clustered at the country level, *** indicates significance at 1\%, ** at 5\%, and * at 10\%. All regressions include country fixed effects and a constant. See appendix for lists of exact countries included in each region. The coefficient estimate on rural population density indicates the value of $\beta$, see equation (\ref{EQ_regress}). Rural population is from HYDE database \citep{hyde31}, and caloric yield is the author's calculations based on the data from \citet{galorozak2016}. The p-value is from a hypothesis test that the estimated $\beta$ is equal to that estimated for Northwest Europe, $\beta^{NWEur}$, and is obtained from an interaction term in a separate regression including both Northwest Europe and the given region, see equation (\ref{EQ_interaction}) and the text for details.
}
\end{table}

\clearpage
\begin{table}[!htb]
\begin{center}
\caption{Estimates of Malthusian Tightness, $\beta$, China, 2000CE}
\label{TAB_beta_china}
{\small
\begin{tabularx}{\textwidth}{lXXXXXX}
\midrule
\multicolumn{7}{l}{Dependent Variable: Log caloric yield ($A_{ic}$)} \\
 & \multicolumn{3}{c}{Province level:} & \multicolumn{3}{c}{District level:} \\ \cmidrule(lr){2-4} \cmidrule(lr){5-7}
 & All& North & South & All   & North   & South \\
 & China & China  & China      & China    & China   & China \\
 & (1) & (2) & (3) & (4) & (5) & (6) \\
\midrule
Rural density 1900  &       0.824&       0.849&       0.104&       0.788&       0.865&       0.171\\
                    &     (0.093)&     (0.103)&     (0.065)&     (0.076)&     (0.090)&     (0.030)\\
\midrule
p-value $\beta=\beta^{North}$&            &            &       0.000&            &            &       0.000\\
Observations        &          30&          15&          15&         329&         156&         173\\
Adjusted R-square   &        0.85&        0.87&        0.10&        0.55&        0.54&        0.43\\

\midrule
\end{tabularx}
}
\end{center}
\vspace{-.5cm}\singlespacing {\footnotesize \textbf{Notes}: Standard errors are clustered at the country level, *** indicates significance at 1\%, ** at 5\%, and * at 10\%. All regressions include country fixed effects and a constant. See appendix for lists of exact countries included in each region. The coefficient estimate on rural population density indicates the value of $\beta$, see equation (\ref{EQ_regress}). Rural population is from HYDE database \citep{hyde31}, and caloric yield is the author's calculations based on the data from \citet{galorozak2016}. The p-value is from a hypothesis test that the estimated $\beta$ is equal to that estimated for North China, $\beta^{North}$, and is obtained from an interaction term in a separate regression including both North and South China, see equation (\ref{EQ_interaction}) and the text for details.
}
\end{table}

\clearpage
\begin{table}[!htb]
\begin{center}
\caption{Estimates of Malthusian Tightness, $\beta$, by Crop Suitability, 2000CE}
\label{TAB_beta_crops}
{\footnotesize
\begin{tabularx}{\textwidth}{lXXXXXX}
\midrule
\multicolumn{7}{l}{Dependent Variable in all panels: Log caloric yield ($A_{ic}$)} \\ \\
\multicolumn{7}{l}{Panel A: Wheat and rice} \\
 & \multicolumn{6}{c}{Inclusion by crop suitability:} \\ \cmidrule(lr){2-7}
 & \multicolumn{4}{c}{Entire world:} & \multicolumn{2}{c}{Ex. Americas:}\\ \cmidrule(lr){2-5} \cmidrule(lr){6-7} 
 & Wheat$>$0& Wheat=0 &         &        & Wheat$>$0   & Wheat=0   \\
 & Rice=0 & Rice$>$0  & Wheat$>$0 & Rice$>$0 & Rice=0    & Rice$>$0   \\
 & (1) & (2) & (3) & (4) & (5) & (6) \\
\midrule
Rural density 1900  &       0.318&       0.085&       0.245&       0.085&       0.318&       0.105\\
                    &     (0.024)&     (0.016)&     (0.033)&     (0.017)&     (0.032)&     (0.028)\\
\midrule
Countries           &         106&          80&         136&         134&          86&          56\\
Observations        &       14792&        9358&       28208&       22774&       11585&        5219\\
Adjusted R-square   &        0.59&        0.24&        0.57&        0.31&        0.61&        0.21\\

\midrule
\\
\multicolumn{7}{l}{Panel B: Tropical crops} \\
                   & \multicolumn{6}{c}{Inclusion by crop suitability:} \\ \cmidrule(lr){2-7}
                   &            &              &          &   Pearl       &  Sweet      & \\
& Cassava$>$0 & Cowpeas$>$0  & Maize$>$0 & Millet$>$0 & Potato$>$0 & Yams$>$0   \\
\midrule
Rural density 1900  &       0.070&       0.120&       0.146&       0.080&       0.105&       0.093\\
                    &     (0.013)&     (0.027)&     (0.027)&     (0.018)&     (0.023)&     (0.020)\\
\midrule
Countries           &          97&         126&         146&          89&         122&         110\\
Observations        &       17104&       22134&       34208&       12821&       22063&       19253\\
Adjusted R-square   &        0.22&        0.43&        0.47&        0.43&        0.35&        0.32\\

\midrule
\\
\multicolumn{7}{l}{Panel C: Temperate crops} \\
                   & \multicolumn{6}{c}{Inclusion by crop suitability:} \\ \cmidrule(lr){2-7}
                   &            & Buck-        &          &          &         & White \\
                   & Barley$>$0 & wheat$>$0  & Oats$>$0 & Flax$>$0 & Rye$>$0 & Potato$>$0   \\
\midrule
Rural density 1900  &       0.245&       0.290&       0.316&       0.285&       0.320&       0.244\\
                    &     (0.033)&     (0.020)&     (0.024)&     (0.020)&     (0.025)&     (0.033)\\
\midrule
Countries           &         136&          84&          74&          81&          74&         133\\
Observations        &       28207&       20925&       19940&       20598&       19981&       27734\\
Adjusted R-square   &        0.57&        0.61&        0.61&        0.62&        0.61&        0.57\\

\midrule
\end{tabularx}
}
\end{center}
\vspace{-.5cm}\singlespacing {\footnotesize \textbf{Notes}: Standard errors are clustered at the country level, *** indicates significance at 1\%, ** at 5\%, and * at 10\%. All regressions include country fixed effects and a constant. The coefficient estimate on rural population density indicates the value of $\beta$, see equation (\ref{EQ_regress}). Rural population is from HYDE database \citep{hyde31}, and caloric yield is the author's calculations based on the data from \citet{galorozak2016}. Inclusion of sub-national units in the regression is based on crop suitability indices from \citet{gaez}, which range from 0 to 100, and are calculated by the author's for each sub-national unit. See text for details.
}
\end{table}

\end{document}