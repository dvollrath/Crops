\documentclass[11pt]{article}
%%%%%%%%%%%%%%%%%%%%%%%%%%%%%%%%%%%%%%%%
\usepackage{amsmath}
\usepackage{verbatim}
\usepackage[usenames,dvipsnames]{color}
\usepackage{setspace}
\usepackage{lscape}
\usepackage{longtable}
\usepackage[top=1.25in,bottom=1.25in,left=1in,right=1in]{geometry}
\usepackage{graphicx}
\usepackage{epstopdf}
\usepackage[usenames,dvipsnames]{pstricks}
\usepackage{epsfig}
\usepackage{pstricks-add}
\usepackage{pst-node}
\usepackage{pst-plot}
\usepackage{fancyhdr}
\usepackage[absolute,showboxes]{textpos}
\usepackage{booktabs}
\usepackage{dcolumn}
\usepackage{arydshln}
\usepackage{natbib}
\usepackage{tabularx}
\usepackage{subfigure}
\usepackage{caption}
\usepackage{minitoc}

\renewcommand{\thetable}{A.\arabic{table}}
\renewcommand{\thesection}{A.\arabic{section}}
\renewcommand{\theequation}{A.\arabic{equation}}
\makeatletter
\renewcommand{\l@section}{\@dottedtocline{1}{1.5em}{2.6em}}
\renewcommand{\l@subsection}{\@dottedtocline{2}{4.0em}{3.6em}}
\renewcommand{\l@subsubsection}{\@dottedtocline{3}{7.4em}{4.5em}}
\makeatother

\newtheorem{proposition}{Proposition}
\newtheorem{corollary}{Corollary}

\setcounter{MaxMatrixCols}{10}
\newcolumntype{d}[1]{D{.}{.}{-2.#1}}
\newenvironment{proof}[1][Proof]{\noindent\textbf{#1.} }{\ \rule{0.5em}{0.5em}}
\setlength{\columnsep}{.2in}
\psset{unit=1cm}

\def\sym#1{\ifmmode^{#1}\else\(^{#1}\)\fi}

%%%%%%%% Create template for table inclusion
\newcommand{\addtab}[4]{
\clearpage

\begin{table}[!htb]
\begin{center}
\caption{#1}
{\footnotesize
\begin{tabularx}{\textwidth}{lXXXXXX}
\midrule
\multicolumn{7}{l}{Dependent Variable in all panels: Log caloric yield ($A_{isg}$)} \\ \\
\multicolumn{7}{l}{Panel A: Regions defined by:} \\ \\
 & \multicolumn{2}{c}{Suitability:} & \multicolumn{2}{c}{Max calories:} & \multicolumn{2}{c}{Harvest area:}\\ \cmidrule(lr){2-3} \cmidrule(lr){4-5} \cmidrule(lr){6-7} 
 & Temperate & Tropical & Temperate  & Tropical  & Temperate  & Tropical \\
 & (1) & (2) & (3) & (4) & (5) & (6) \\
\midrule
\input{#3}
\midrule
\\
\multicolumn{7}{l}{Panel B: With other restrictions (using suitability to define temperate/tropical)} \\ \\
 & \multicolumn{2}{c}{Urban Pop. $<25K$:} & \multicolumn{2}{c}{Ex. Europe/N. Amer.:} & \multicolumn{2}{c}{Rural dens. $>$ 25th P'tile:}\\ \cmidrule(lr){2-3} \cmidrule(lr){4-5} \cmidrule(lr){6-7}
 & Temperate & Tropical & Temperate  & Tropical  & Temperate  & Tropical \\
 & (1) & (2) & (3) & (4) & (5) & (6) \\
\midrule
\input{#4}
\midrule
\end{tabularx}
}
\end{center}
\vspace{-.5cm}\singlespacing {\footnotesize #2
}
\end{table}
}
%%%%%%%%%%%%%%%%%%%%%%%%%%%%



\begin{document}
\begin{titlepage}
\hfill \textsc{FOR ONLINE PUBLICATION}

\vspace{1in} \noindent {\large \today}

\vspace{.5in} \noindent {\Large \textbf{\strut The role of land in temperate and tropical agriculture}}

\vspace{.25in} \noindent {\large T. Ryan Johnson}

\vspace{.05in} \noindent Washington University

\vspace{.25in} \noindent {\large Dietrich Vollrath}

\vspace{.05in} \noindent University of Houston

\vspace{2in} \noindent \textsc{Online Appendix} \hrulefill

\vspace{.05in} \noindent Robustness checks and alternative assumptions for empirical work from the main paper are contained here. Also included are definitions of countries included in regions used in paper, as well as additional theoretical work related to the model.
\vspace{.1in} \hrule

\end{titlepage}

\pagebreak 

\tableofcontents

\section{General version of empirical setup}
This is to demonstrate that the elasticity of agricultural productivity with respect to the density of agricultural labor is equal to the elasticity of agricultural output with respect to land given any constant returns to scale production function. Let agricultural production be
\begin{equation}
    Y_i = A_i F(X_i,K_i,L_{Ai}) 
\end{equation}
for district $i$ in province $I$, where $F()$ is a constant returns to scale function with respect to the three inputs: land, capital, and labor. As in the main paper, we make the same mobility assumptions for labor and capital across districts, which again implies that $K_i/L_{Ai}$ is identical for all districts. 

Dividing the production function through by $L_i$, and multiplying by $\phi_L$ (the share of output paid to labor), we have that
\begin{equation}
    \phi_L Y_i/L_{Ai} = \phi_L A_i F(X_i/L_i,K_i/L_i,1).
\end{equation}
Define $x_i = X_i/L_{Ai}$ and $k_i = K_i/L_{Ai}$ as the per-worker amounts of land and capital, and define
\begin{equation}
    f(x_i,k_i) = F(X_i/L_i,K_i/L_i,1)
\end{equation}
as the intensive form of the aggregate production function. By the mobility assumption, we know that $\phi_L Y_i/L_{iA} = w$, where the wage is common across districts. This allows us to write that 
\begin{equation}
    w = \phi_L A_i f(x_i,k_i).
\end{equation}

Holding this wage constant as it is given by province-level factors, and noting that $k_i$ is also given by province-level factors, we use the implicit function theorem to solve for
\begin{equation}
    \frac{\partial A_i}{\partial x_i} \frac{x_i}{A_i} = - \frac{\phi_L A_i f_1(x_i,k_i)}{\phi_L f(x_i,k_i)}\frac{x_i}{A_i} = - \frac{f_1(x_i,k_i) x_i}{f(x_i,k_i)}.
\end{equation}
The elasticity of productivity, $A_i$, with respect to land per worker, $x_i$, is equal to the elasticity of $f()$ with respect to land per worker. This simply implies that the relationship of land per worker to productivity depends on how sensitive output per worker is to land per worker. 

It is straightforward to show, given the constant returns to scale production function, that
\begin{equation}
    \frac{f_1(x_i,k_i) x_i}{f(x_i,k_i)} = \frac{F_1(X_i,K_i,L_i) X_i}{F(X_i,K_i,L_i)}
\end{equation}
or that the elasticity of the intensive production function with respect to land per worker is equal to the elasticity of the aggregate production function with respect to land. Thus it follows that 
\begin{equation}
    \frac{\partial A_i}{\partial x_i} \frac{x_i}{A_i} = - \frac{F_1(X_i,K_i,L_i) X_i}{F(X_i,K_i,L_i)}
\end{equation}
or that the elasticity of productivity with respect to land per worker, $x_i$, is equal to the negative of the elasticity of aggregate output with respect to land. It is then trivial that the elasticity of productivity with respect to density, $1/x_i$, is equal to the elasticity of aggregate output with respect to land. The Cobb-Douglas assumption used in the main paper is not necessary to derive the main estimating equation used in the paper.

\section{Solving for Labor Share and Real Income}\label{APP_solve}
In Section 4 of the main paper we solved for $L_A/L$ and $y$, the agricultural labor share and real income, respectively. The algebra leading to equations (12) and (13) in the main paper is as follows. From here forward, any equation references without a prefix, (e.g. (2)), refer to the main paper, while references with a prefix (e.g. (A.2)) refer to this Appendix.

Based on the district-level production functions from (1), total agricultural supply in province $I$ can be written as
\begin{equation}
Y_A = \sum_{i \in I} A_{i} X_{i}^{\beta} \left(K_{Ai}^{\alpha}L_{Ai}^{1-\alpha}\right)^{1-\beta}. \label{EQ_caL}
\end{equation}
We know each $L_{Ai}$ from (4). By a similar logic used for labor we can establish that the allocation of capital to any individual location $i$ is
\begin{equation}
    K_{Ai} = A_{i}^{1/\beta} X_i \frac{K_A}{\sum_{j\in I} A_{j}^{1/\beta}X_{j}} \label{EQ_KaX}
\end{equation}
where $K_A$ is the aggregate allocation of capital to agriculture. Combine (\ref{EQ_KaX}) with the expression in (\ref{EQ_caL}) and we can solve for 
\begin{equation}
    Y_A = A_A \left(\frac{K_A}{L_A}\right)^{\alpha(1-\beta)} L_A^{1-\beta} \nonumber
\end{equation}
where 
\begin{equation}
    A_A = \left(\sum_{j\in I} A_{j}^{1/\beta}X_{j} \right)^\beta \nonumber
\end{equation}
is the measure of aggregate agricultural total factor productivity for the province. 

With the assumption that land earns no return, and the share earned by capital is $\phi_K$ is both sectors, and for labor the share is $\phi_L$ in both sectors, it follows that the capital/labor ratio in both sectors is equal to the aggregate capital labor ratio,
\begin{equation}
    \frac{K_A}{L_A} = \frac{K_N}{L_N} = \frac{K}{L} = \frac{w}{r}\frac{\phi_K}{\phi_L}. \nonumber
\end{equation}

Using the equilibrium condition on wages across sectors from (9), we can solve for 
\begin{equation}
    \frac{p_A}{p_N} = \frac{Y_N}{L_N}\frac{L_A}{Y_A}.\label{EQ_papn}
\end{equation}
Noting that $Y_N = c_N L$ and $Y_A = c_A L$, we can rearrange this be
\begin{equation}
    \frac{p_A c_A}{p_N c_N} = \frac{L_A}{L_N}, \label{EQ_expend}
\end{equation}
which shows that the relative amount of labor employed in agriculture and non-agriculture is equal to the relative expenditures on those goods. With the adding up conditions $L_A + L_N = L$ and $p_Ac_A + p_N c_N = M$, it follows that in log terms
\begin{equation}
    \ln L_A/L = \ln p_A c_A/M. \label{EQ_pacaM}
\end{equation}

Turning to the demand function from (8), we can re-arrange that to
\begin{equation}
   (1-\epsilon) \ln p_A c_A/M = \ln \theta_A  + (\epsilon-\gamma)(\ln p_N - \ln p_A) - \epsilon \ln c_A. \nonumber
\end{equation}
Using the relationships in (\ref{EQ_papn}) and (\ref{EQ_pacaM}), as well as the fact that $c_A = (Y_A/L_A)(L_A/L)$, we can substitute here to find
\begin{equation}
    (1-\epsilon)\ln L_A/L = \ln \theta_A + (\epsilon-\gamma)\left(\ln Y_N/L_N - \ln Y_A/L_A \right) - \epsilon \left(\ln Y_A/L_A + \ln L_A/L \right). \nonumber
\end{equation}
Collecting terms we have
\begin{equation}
    \ln L_A/L = \ln \theta_A + (\epsilon-\gamma) \ln Y_N/L_N - \gamma \ln Y_A/L_A. \nonumber
\end{equation}
Using the production functions in (7) and (\ref{EQ_caL}), we can write this as
\begin{equation}
    \ln L_A/L = \ln \theta_A + (\epsilon-\gamma) \ln \left(A_N (K/L)^{\alpha}\right) - \gamma \ln \left(A_A (K/L)^{\alpha(1-\beta)} L_A^{-\beta}\right) - \gamma \beta \ln L + \gamma \beta \ln L, \nonumber
\end{equation}
where we've added and subtracted the term involving $L$. At this point, what remains is to separate the productivity and capital terms using the logs, and then straightforward algebra to arrive at
\begin{equation}
    \ln L_A/L = \ln \theta_A + \frac{\beta\gamma}{1-\beta\gamma} \ln L - \frac{\gamma}{1-\beta\gamma} \ln A_A + \frac{\gamma - \epsilon}{1-\beta\gamma} \ln A_N + \frac{\alpha(\beta\gamma - \epsilon)}{1-\beta\gamma} \ln K/L. \nonumber
\end{equation}
Exponentiating this, we arrive at (10) from the main text.

For real income, in agricultural terms we have
\begin{equation}
    y = \frac{M}{p_A} = c_A + \frac{p_N}{p_A} c_N. \nonumber
\end{equation}
Using (\ref{EQ_expend}) we can write this as
\begin{equation}
    y = c_A + \frac{p_N c_N}{p_A c_A}c_A = c_A \left(1 + \frac{L_N}{L_A}\right) = c_A \frac{L}{L_A}. \nonumber
\end{equation}
Noting that $c_A = Y_A/L$, we have that
\begin{equation}
    y = \frac{Y_A}{L_A} = A_A (K/L)^{\alpha(1-\beta)} (L_A/L)^{-\beta} L^{-\beta}, \nonumber
\end{equation}
where the second equality follows from (\ref{EQ_caL}). At this point, we can use (10) to plug in for $L_A/L$ in the above equation, and solve for 
\begin{equation}
    \ln y = \frac{1}{1-\beta\gamma} \ln A_A - \frac{\beta}{1-\beta\gamma} \ln L + \frac{\beta(\epsilon-\gamma)}{1-\beta\gamma} \ln A_N + \frac{\alpha(1-\beta) + \alpha\beta(\epsilon-\gamma)}{1-\beta\gamma} \ln K/L. \nonumber
\end{equation}
Exponentiating, we arrive at (11) in the main text.

\section{Alternative Mobility Assumptions}
Our baseline specifications are built off of a model that assumes agricultural output, as well as labor and capital, are freely mobile across districts within a given province (although we do not require them to be mobile across provinces). However, even within a given province, there may be frictions or limits on the mobility of either output or inputs, or both. If these frictions exist, then our regressions may not be delivering unbiased estimates of $\beta$. We outline two alternative assumptions about mobility here, and how our results relate to those. 

\subsection{Immobile Factors}
The baseline model assumes capital and labor are free to move between districts within a region. If we make factors immobile, but allow both agricultural and non-agricultural output to move between districts, this changes the specification of the relationship between agricultural productivity and rural density. 

The agricultural production function for a district is the same as in (1), and we also need to specify a production function for non-agriculture. We do so as $Y_{Ni} = A_{Ni} K_{Ni}^{\alpha} L_{Ni}^{1-\alpha}$. Capital and labor are assumed to be mobile \textit{within} the district between the two sectors, implying that the return to capital and the return to labor are equalized across different uses. Because of this, the capital/labor ratio in both sectors will be identical, with $K_{Ai}/L_{Ai} = K_{Ni}/L_{Ni} = K_i/L_i$, where $K_i/L_i$ is the district's aggregate capital/labor ratio. 

Equality of the return to labor across different sectors implies that 
\begin{equation}
    p_A (1-\alpha)(1-\beta)\frac{Y_{Ai}}{L_{Ai}} = p_N (1-\alpha)\frac{Y_{Ni}}{L_{Ni}}. \nonumber
\end{equation}
Using the condition that the capital/labor ratio will be identical across the two sectors, and re-arranging this relationship, we have that
\begin{equation}
    p_A (1-\beta) A_{Ai} \left(\frac{K_i}{L_i}\right)^{\alpha(1-\beta)} \left(\frac{X_i}{L_{Ai}}\right)^{\beta} = p_N A_{Ni} \left(\frac{K_i}{L_i}\right)^{\alpha} \nonumber
\end{equation}
Taking logs, are again re-arranging terms, we arrive at
\begin{equation}
    \ln A_{Ai} = \beta \ln L_{Ai}/X_i + \ln A_{Ni} + \alpha\beta \ln K_i/L_i + \ln p_N/p_A. \nonumber
\end{equation}
This equation shows that the relationship between agricultural productivity, $A_{Ai}$, and rural density, $L_{Ai}/X_i$, can still be used to recover an estimate of $\beta$. To do this, we must control for the district-specific levels of non-agricultural productivity, $A_{Ni}$, and capital/labor, $K_i/L_i$. While we do not have direct measures of those, we believe that our control for night lights will act as a decent proxy for these terms. Finally, the price ratio, $p_N/p_A$, is the province relative price, as goods are traded freely, so this will be captured by the province level fixed effects.

If our night lights control is not capturing the variation in $A_{Ni}$ or $K_i/L_i$, then our estimates may be biased if there is a relationship between those variables and rural density. In particular, if rural density is negatively related to $A_{Ni}$ and/or $K_i/L_i$ then we could be under-stating the value of $\beta$. It is not clear why this negative relationship would hold only in tropical areas (with small estimate $\beta$ values), but not in other areas.

\subsection{Autarkic Districts}
If districts are entirely closed, in that neither factors of production nor output can move between districts, then this again changes the specification of our regressions. Here, the crucial remaining assumption is that the value of $\beta$ is the same across all districts within a given province. 

Within each district, let the amount of agricultural output consumed be $c_{Ai}$, and hence market clearing within the district requires $c_{Ai} L_i = Y_i$ for agricultural output. Using the same production function as in the main section, and again assuming that capital and labor move freely between sectors (non-agriculture and agriculture) so that the capital/labor ratios are equal to the aggregate ratio, we have
\begin{equation}
    c_{Ai} L_i = A_i X_i^{\beta} \left(K_i/L_i\right)^{\alpha(1-\beta)} L_{Ai}^{1-\beta}. \nonumber
\end{equation}
Taking logs and re-arranging, we have the following
\begin{equation}
    \ln A_i = \beta \ln L_{Ai}/X_i - \ln L_{A_i}/L_i - \alpha(1-\beta) \ln K_{i}/L_{i} + \ln c_{Ai}. \nonumber
\end{equation}
We can recover an estimate of $\beta$ from the relationship of productivity and rural density, but now must control for the agricultural share of labor, $L_{Ai}/L_i$, the capital/labor ratio, and the consumption of agricultural goods per capita. For $L_{Ai}/L_i$, we have this data, and can include it directly in a regression (it is implicitly included in our baseline regression when we use the percent urban). For the capital/labor ratio and consumption of agricultural goods, we believe that the night lights data are a decent proxy for these terms.

Including the log of $L_{Ai}/L_i$ explicitly as a control is possible, and the results are almost identical to our baseline results. This may still be biased, however, if the night lights proxy does not pick up the variation in consumption or the capital/labor ratio. If the capital/labor ratio is positively related to the rural density, then we would be under-estimating the true value of $\beta$. The small estimated values of $\beta$ in tropical areas may be because of this relationship, although it is not clear why rural density would be positively related to capital/labor ratios only in tropical areas. Alternatively, if consumption of agricultural goods is negatively related to rural density, and we are not controlling for it with night lights, then we may be under-estimating $\beta$. This could possibly be true only in tropical areas if they are relatively poor, whereas this relationship no longer holds in richer, temperate areas. This is clearly a possibility, although recall that this would only be a problem if we believe that districts are \textit{autarkic}, which may be an extreme assumption.

\section{Adding Malthusian fertility responses}
Our main model is static, taking the size of population (and capital) as given. By adding in a simple Malthusian fertility response, one can establish several results related to population density and the land elasticity. 

Without specifying a particular utility function, let population growth from $t$ to $t+1$ be a function of income per capita in $t$
\begin{equation}
    n_{t+1} = y_t^{\theta}
\end{equation}
where $0<\theta<1$ so that population growth is a concave function of income per capita. This delivers, using equation (11) from the main text, 
\begin{equation}
    L_{t+1} = n_{t+1} L_t = \left(A_A A_N^{\beta(\epsilon-\gamma)}\hat{k}^{\Omega}\right)^{\frac{\theta}{1-\beta\gamma}} L_t^{\frac{1-\beta(\theta+\gamma)}{1-\beta\gamma}}.
\end{equation}
Examining the exponent on $L_t$, it is clear that this is less than one, making $L_{t+1}$ a concave function of $L_t$, and thus we have a stable steady state for population. Solving for that steady state by setting $L_{t+1} = L_t$ we have
\begin{equation}
    L^{\ast} = \left(A_A A_N^{\beta(\epsilon-\gamma)}\hat{k}^{\Omega}\right)^{\frac{1}{\beta}}.
\end{equation}
From this, one can see the influence of agricultural productivity on population, and hence on population density. Note that the elasticity of steady state population with respect to $A_A$ depends inversely on $\beta$. As the land elasticity gets larger, the effect of agricultural productivity on population size decreases. The positive Malthusian relationship of population size and (agricultural) productivity remains, but because the Malthusian constraint is much tighter when $\beta$ is large, the relationship is not as strong as when $\beta$ is small. 

\section{Alternative Population Data}
As mentioned in the main text, given possible issues with the HYDE data on population, we use two alternative sources of population data.

\subsection{GRUMP Data}
We accessed both the gridded population map, and the urban extents grid. The GRUMP data is provided at 30 arc-second grids (rougly 1km squares), a higher resolution than the HYDE data (which has 5 arc-minute grids, or roughly 10km squares). We overlay the urban extents grid on the population map, and retrieve only the population count of grid cells that are not part of an urban extent. We then sum up the population count of grid cells within each district. The district definitions are from GADM, identical to those used with the HYDE data, so we can compare the counts directly.

Because GRUMP counts zero rural residents in an area that is part of an urban extent, and all population in non-urban locations as rural, the variation in rural residents across districts is more severe than with HYDE. Some districts in GRUMP are entirely covered by urban extents, and so have zero rural residents. Hence the GRUMP data leads to 28,471 districts with data on rural density, compared to 35,451 using HYDE. For those 28,471 districts, the correlation of (log) rural density across the two datasets is 0.81, significant at less than 1\%. 

\subsection{IPUMS Data}
We use 39 countries that have both geographic identifier data (the GEOLEV2 variable from IPUMS) as well as information on individual industry of employment. We create a 0/1 variable indicating whether an individual is an agricultural worker (meaning they are reported as in the workforce). We then aggregate this variable (weighted by their IPUMS provided sampling weight) across individuals within a geographic area to get a count of the total agricultural workers. Using a similar method, we are also able to count the number of urban residents, which allows us to measure the percent urban within a geographic area. We end up with a total of 8,393 geographic areas.

Before we run regressions, the IPUMS data is useful in assessing how good of an approximation rural population (including workers and non-workers) is for the number of agricultural workers. The correlation of (log) rural residents and (log) agricultural workers across the areas is 0.91, significant at less than 1\%. There are a few outliers where the number of agricultural workers is high relative to rural residents, which likely represents agricultural processing work in urban areas, or urban farmers with small plots. Our results are not affected by excluding these outliers.

The geographic areas provided by IPUMS in the GEOLEV2 variable are in many cases agglomerations of the districts we use from GADM. This is because IPUMS aggregates districts with fewer than 25000 observations (to protect anonymity) or districts whose boundaries have changed over time (so that the agglomerations are comparable over time for a given country). This means the IPUMS geographic areas are not directly comparable to our districts. Because the IPUMS agglomerations are much larger than districts, it is not practical to use province/state fixed effects, as most of these have only one or two IPUMS areas within them. Hence we run our regressions only with country fixed effects. Because the GEOLEV2 areas are different than the districts in our baseline specifications, we create new GEOLEV2 level versions of our caloric suitability index, night lights data, and other crop suitability measures. 

The 39 countries included from IPUMS are, with the census date listed: 
Argentina (2001), Austria (2001), Bolivia (2001), Brazil (2000), Cambodia (1998), Cameroon (2005), Chile (2002), Colombia (2005), Costa Rica (2000), Ecuador (2001), El Salvador (2007), Fiji (1996), Ghana (2000), Greece (2001), Haiti (2003), India (1999), Iran (2006), Iraq (1997), Jordan (2004), Kyrghzstan (1999), Malawi (1998), Mexico (2000), Morocco (2004), Mozambique (1997), Panama (2000), Peru (2007), Sierra Leone (2004), South Africa (2001), Spain (2001), South Sudan (2008), Sudan (2008), Turkey (2000), Uganda (2002), Egypt (1996), Tanzania (2002), United States (2000), Burkina Faso (1996), Venezuela (2001), Zambia (2000) 

\section{Alternative measure of $A_{isg}$}
The CSI index used as the baseline measure of $A_{isg}$ combines the raw potential tonnage of production of specific crops with information on their calorie content so that one can compare the \textit{caloric} yield of each crop within a given grid-cell. Then the maximum value of that caloric yield is selected across crops, and those maximums are aggregated across grid-cells in a district to arrive at the $A_{isg}$ measure. This follows Galor and Ozak's (2016) methodology, but there may be an issue with using these calorie counts to compare crops. The calorie count of each crop may not be an accurate measure of the available calories from those crops, given storage and preparation techniques. A worry is that we may have created variation in $A_{isg}$ because of variation in the calorie counts of crops, and that this is driving the results. For example, given paddy rice's very high caloric yield in the Galor and Ozak methodology, it is possible that we are overstating the productivity of districts that are in fact very un-productive from the perspective of farmers, but because they are capable of growing rice, Galor and Ozak have coded them as having very high productivity. This would bias our estimates of $\beta$ down for these areas. 

To see that this is not driving our results, we have performed separate regressions estiamting $\beta$ where we use a single crop-specific raw yield from GAEZ as the measure of productivity (e.g. wheat or rice). In this case, there is no caloric information employed at all, as we are not trying to combine data across crops. Each district is thus measured on a comparable basis. Table \ref{TAB_indcrop} shows these results. Panel A is for the temperate districts identified in the main paper as those capable of growing the temperate crops, while Panel B is for the tropical districts capable of growing tropical crops. In both panels, the first column replicates our baseline results from Table 2 of the main paper.

For temperate crops, the baseline estimate of $\beta$ is 0.228. The next six columns show the estimated value of $\beta$ if in place of the CSI yield from Galor and Ozak as the $A_{isg}$ variable, we use the raw yield of the specified crop. For example, using just the raw yield of barley to measure $A_{isg}$ in temperate districts, we find an estimated $\beta$ of 0.225. Given that this is nearly identical to the baseline estimate, this indicates that the baseline is not driven simply by the caloric values assigned to barley versus other temperate crops. The rest of the columns show the same kind of result. In each case, the estimate of $\beta$ is very close to 0.228, indicating that the CSI index is not driven by caloric information, but by common variation in the raw productivity of these crops across districts. In Panel B, a similar story is shown. The baseline estimate for the tropical districts is 0.132, while each of the separate columns delivers a result nearly identical, save pearl millet (although still at 0.145). Again, the baseline estimate using the CSI index is not driven simply by the use of calories to weight the different crops.

The implication of these results is that \textit{any} scheme used to weight raw yields across crops is going to deliver similar results regarding $\beta$. Prices, or alternative means of measuring nutritional quality, if used to construct $A_{isg}$ would still show that temperate areas have larger land elasticities than tropical areas.

\section{Political region results}
Table \ref{TAB_beta_subregion} shows results grouping districts by their ``macro-region''. Within each of these regions, all districts are assumed to have identical values of $\beta$. Given the rough correlation of these regions with different climate types, the pattern of results suggest similar results to the baseline. The final panel shows separate results for China (separated into a temperate north and tropical south), Japan, and Korea. 

\noindent\textbf{Regions:} Countries are included as follows:. 
\begin{itemize}
    \setlength\itemsep{0pt}
    \item \textbf{Central and West Asia}: Afghanistan, Azerbaijan, Bhutan, Georgia, Iran, Iraq, Jordan, Kazakhstan, Kyrgyzstan, Lebanon, Oman, Pakistan, Palestina, Russia (Asia), Syria, Tajikistan, Turkey, Uzbekistan
\item \textbf{Eastern Europe}: Belarus, Bulgaria, Czech Republic, Hungary, Poland, Romania, Russia (Europe), Slovakia, Ukraine
\item \textbf{North Africa}: Algeria, Egypt, Morocco, Sudan, Tunisia
\item \textbf{Northwest Europe}: Austria, Belgium, Denmark, Estonia, Finland, France, Germany, Isle of Man, Latvia, Lithuania, Luxembourg, Netherlands, Norway, Sweden, Switzerland, United Kingdom
\item \textbf{South Africa}: Botswana, Namibia, South Africa, Swaziland
\item \textbf{South and Southeast Asia}: Bangladesh, Brunei, Cambodia, India, Indonesia, Laos, Malaysia, Myanmar, Philippines, Sri Lanka, Thailand, Timor-Leste, Vietnam
\item \textbf{Southern Europe}: Albania, Bosnia and Herzegovina, Croatia, Italy, Portugal, Serbia, Slovenia, Spain
\item \textbf{Temperate Americas}: Argentina, Canada, Chile, United States, Uruguay
\item \textbf{Tropical Africa}: Angola, Benin, Burkina Faso, Burundi, Cameroon, Central African Republic, Chad, Côte d'Ivoire, Democratic Republic of the Congo, Equatorial Guinea, Eritrea, Ethiopia, Gabon, Gambia, Ghana, Guinea, Guinea-Bissau, Kenya, Liberia, Madagascar, Malawi, Mali, Mauritania, Mozambique, Niger, Nigeria, Republic of Congo, Reunion, Rwanda, Senegal, Sierra Leone, Somalia, South Sudan, São Tomé and Príncipe, Tanzania, Togo, Uganda, Zambia, Zimbabwe
\item \textbf{Tropical Americas}: Bolivia, Brazil, Colombia, Costa Rica, Cuba, Dominican Republic, Ecuador, El Salvador, French Guiana, Guadeloupe, Guatemala, Guyana, Haiti, Honduras, Martinique, Mexico, Nicaragua, Panama, Paraguay, Peru, Suriname, Venezuela

\end{itemize}

\noindent\textbf{For China-only regressions:} We exclude Tibet, Xinjiang, Gansu, and Qinghai entirely, given that their climates do not fit well into the temperate versus sub-tropical distinction we make in the regressions.
\begin{itemize}
    \setlength\itemsep{0pt}
    \item \textbf{Temperate provinces}: Hebei, Heilongjiang, Jilin, Liaoning, Nei Mongol, Ningxia Hui, Shaanxi, Shanxi, Tianjin, Sichuan, Shandong, Yunnan
    \item \textbf{Sub-tropical provinces}: Guangxi, Guangdong, Fujian, Jiangxi, Hunan, Guizhou, Chongqing, Hubei, Anhui, Zhejiang, Henan, Jiangsu, Hainan
\end{itemize}

\noindent\textbf{Russian provinces:} We split Russia into separate Asian and European sections for inclusion in the regions. That breakdown takes place at the province level
\begin{itemize}
    \setlength\itemsep{0pt}
    \item \textbf{Russia(Asia)}: Altay, Amur, Buryat, Chelyabinsk, Gorno-Altay, Irkutsk, Kemerovo, Khabarovsk, Khakass, Khanty-Mansiy, Krasnoyarsk, Kurgan, Novosibirsk, Omsk, Primor'ye, Sakhalin, Sverdlovsk, Tomsk, Tuva, Tyumen', Yevrey, Zabaykal'ye
\item \textbf{Russia(Europe)}: Adygey, Arkhangel'sk, Bashkortostan, Belgorod, Bryansk, Chechnya, Chuvash, Dagestan, Ingush, Ivanovo, Kabardin-Balkar, Kaliningrad, Kalmyk, Kaluga, Karachay-Cherkess, Kirov, Komi, Kostroma, Krasnodar, Kursk, Leningrad, Lipetsk, Mariy-El, Mordovia, Moskva, Nizhegorod, North Ossetia, Novgorod, Orel, Orenburg, Penza, Perm', Pskov, Rostov, Ryazan', Samara, Saratov, Smolensk, Stavropol', Tambov, Tatarstan, Tula, Tver', Udmurt, Ul'yanovsk, Vladimir, Volgograd, Vologda, Voronezh, Yaroslavl'

\end{itemize}

\section{Climate zone results}
For this, we use the K{\"o}ppen-Geiger scheme, which classifies each grid cell on the planet on three dimensions. First are the \textit{main climate} zones: equatorial (denoted with an ``A''), arid (B), warm temperate (C), and snow (D).\footnote{There is another classification of climate, polar (E), but that covers only areas that are uninhabited for all intents and purposes.} Second, each grid-cell has a \textit{precipitation} classification: fully humid (f), dry summers (s), dry winters (w), monsoonal (m), desert (D), and steppe (S). Finally, there is the \textit{temperature} dimension: hot summers (a), warm summers (b), cool summers (c), hot arid (h), and dry arid (k).\footnote{There are three other temperature classifications - extreme continental, polar frost, and polar tundra - that also cover only uninhabited areas.} Each grid cell thus receives either a three or two-part code. The area around Paris, for example, is ``Cfb'', meaning it is a warm temperate area, fully humid (rain throughout the year), with warm summers. The area near Saigon is ``Aw'', meaning it is equatorial, with dry winters. There is no separate temperature dimension assigned to equatorial zones, as it tends to be redundant.

What we do in Table \ref{TAB_beta_kg} is divide districts into regions based on their K{\"o}ppen-Geiger classifications, as opposed to crop suitability or production data. We do this along each individual dimension (climate, precipitation, and temperature), including a district in a region if more than 66\% of its land area falls in the given zone. For example, for the equatorial region, we include all districts in which 66\% (or more) of their land area is classified as being in ``A'' in the K{\"o}ppen-Geiger system, regardless of their precipitation or temperature codes. Narrowing down to very specific classifications (``Cfb'', for example) is impractical because the number of districts becomes very small. Similar to the temperate/tropical regressions, the climate zone regions do not force heterogeneous districts to be lumped into single regions based on their nation. 

\section{Expanded samples}
Our baseline samples are Tropical and Temperate. As defined in the main text, the Tropical sample includes districts that are suitable for growing specific crops (cassava, cowpeas, pearl millet, sweet potato, wet rice, yams) but have zero suitability for growing others (barley, buckwheat, oats, rye, white potato, wheat). Temperate is defined in the reverse manner. These definitions exclude 15,692 districts that are capable of growing \textit{both} types of crops. 

Column (1) of Table \ref{TAB_beta_expand} shows the results for just those districts suitable for growing both kinds of crops. The estimated $\beta$ is 0.140, roughly in line with the result for the Tropical sample. Column (2) differs by including any districts that are suitable for Temperate crops (regardless of their suitability for growing Tropical crops). This gives a result of 0.180, smaller than our baseline estimate where we focus on districts suitable only for Temperate crops. Column (3) estimates $\beta$ for any districts that are capable of growing Tropical crops, regardless of their ability to grow Temperate, and the value is 0.132. Again, we see a difference between these two types of regions. However, the gap is smaller, consistent with the fact that we are not distinguishing them as clearly, and the fact that columns (2) and (3) include the 15,692 districts that can grow both kinds of crops. Columns (4)-(6) repeat these regressions, only limiting the sample by excluding districts with large urban areas. We find similar results in this case.

\section{Robustness Tables}
\listoftables

\clearpage
\begin{table}[!htb]
\begin{center}
\caption{Estimates of $\beta$ using individual crop productivity terms}
\label{TAB_indcrop}
{\footnotesize
\begin{tabularx}{\textwidth}{lXXXXXXX}
\midrule
\multicolumn{8}{l}{Panel A: Using only temperate districts defined by crop suitability} \\ \\
\multicolumn{8}{c}{Dependent variable is $A_{isg}$ measured by:} \\ \\
 & CSI & Barley & Buckwheat & Oats & Rye & W. Pot. & Wheat \\
 & (1) & (2) & (3) & (4) & (5) & (6) & (7) \\
\midrule
Log rural density   &       0.228&       0.225&       0.218&       0.222&       0.212&       0.233&       0.227\\
                    &     (0.021)&     (0.019)&     (0.023)&     (0.027)&     (0.021)&     (0.029)&     (0.019)\\
\midrule
p-value $\beta=0$   &       0.000&       0.000&       0.000&       0.000&       0.000&       0.000&       0.000\\
Countries           &          91&          91&          69&          68&          68&          91&          91\\
Observations        &       10661&       10628&        9699&        9834&        9804&       10597&       10631\\
Adjusted R-square   &        0.24&        0.21&        0.18&        0.23&        0.21&        0.23&        0.22\\

\midrule
\\
\multicolumn{8}{l}{Panel B: Using only tropical districts defined by crop suitability} \\ \\
\multicolumn{8}{c}{Dependent variable is $A_{isg}$ measured by:} \\ \\
 & CSI & Cassava & Cowpea & P. Millet & Sw. Pot. & Wet Rice & Yams \\
 & (1) & (2) & (3) & (4) & (5) & (6) & (7) \\
\midrule
Log rural density   &       0.132&       0.137&       0.137&       0.145&       0.136&       0.134&       0.133\\
                    &     (0.018)&     (0.022)&     (0.019)&     (0.018)&     (0.019)&     (0.028)&     (0.020)\\
\midrule
p-value $\beta=0$   &       0.000&       0.000&       0.000&       0.000&       0.000&       0.000&       0.000\\
Countries           &          81&          76&          80&          79&          79&          75&          79\\
Observations        &        9088&        8843&        9074&        8265&        9066&        8448&        9020\\
Adjusted R-square   &        0.12&        0.10&        0.11&        0.07&        0.11&        0.05&        0.10\\

\midrule
\end{tabularx}
}
\end{center}
\vspace{-.5cm}\singlespacing {\footnotesize The panels differ in the districts included in each regression. In Panel A, only districts that are suitable for temperate agriculture are inclueded (based on the criteria we outline in the paper based on GAEZ suitability measures). In Panel B, on tropical districts are included. The columns differ by the variable used to measure $A_{isg}$, inherent agricultural productivity. Column (1) uses the CSI index from Galor and Ozak (2016), as in our baseline results. Columns (2)-(7) use the raw potential yield (in tonnes) of the crop, from the GAEZ, for the crop specified. Additional controls are as in our basline results, and include province fixed effects. Conley standard errors, adjusted for spatial auto-correlation, are shown in parentheses. 
}
\end{table}

\begin{table}[!htb]
\begin{center}
\caption{Estimates of Land Elasticity, $\beta$, by K{\"o}ppen-Geiger Zone, 2000CE}
\label{TAB_beta_kg}
{\footnotesize
\begin{tabularx}{\textwidth}{lXXXXXX}
\midrule
\multicolumn{7}{l}{Dependent Variable in all panels: Log caloric yield ($A_{isg}$)} \\ \\
\multicolumn{7}{l}{Panel A: Climate Zones} \\
 & Equatorial & Arid & Temperate & Snow  &     &   \\
 & (1) & (2) & (3) & (4) &  & \\
\midrule
Log rural density   &       0.111&       0.154&       0.169&       0.230\\
                    &     (0.015)&     (0.026)&     (0.017)&     (0.026)\\
\midrule
p-value $\beta=0$   &       0.000&       0.000&       0.000&       0.000\\
p-value $\beta=\beta^{Equa}$&            &       0.151&       0.007&       0.000\\
Countries           &          79&          56&          94&          40\\
Observations        &       11461&        2822&       13717&        6327\\
Adjusted R-square   &        0.11&        0.10&        0.15&        0.19\\

\midrule
\\
\multicolumn{7}{l}{Panel B: Precipitation Zones} \\
& Fully     & Dry         & Dry        &              &            & \\
& Humid & Summer & Winter & Monsoon & Desert & Steppe \\
 & (1) & (2) & (3) & (4) & (5) & (6) \\
\midrule
Log labor/land ratio ($\beta_g$)&       0.240&       0.215&       0.124&       0.125&       0.130&       0.147\\
                    &     (0.044)&     (0.063)&     (0.022)&     (0.039)&     (0.072)&     (0.029)\\
\midrule
p-value $\beta=0$   &       0.000&       0.001&       0.000&       0.001&       0.072&       0.000\\
p-value $\beta=\beta_{Humid}$&            &       0.739&       0.020&       0.043&       0.190&       0.072\\
Countries           &          78&          37&          67&          32&          20&          49\\
Observations        &       13545&        2373&        7695&        1267&         146&        1735\\
R-square (ex. FE)   &        0.17&        0.17&        0.15&        0.17&        0.17&        0.16\\

\midrule
\\
\multicolumn{7}{l}{Panel C: Temperature Zones} \\
    & Hot        & Warm        & Cool       & Hot      & Cold     &  \\
    & Summer & Summer & Summer & Arid & Arid &   \\
 & (1) & (2) & (3) & (4) & (5) &  \\    
\midrule
Log labor/land ratio ($\beta_g$)&       0.142&       0.226&       0.207&       0.140&       0.190\\
                    &     (0.021)&     (0.053)&     (0.076)&     (0.035)&     (0.046)\\
\midrule
p-value $\beta=0$   &       0.000&       0.000&       0.007&       0.000&       0.000\\
p-value $\beta=\beta_{Humid}$&            &       0.065&       0.405&       0.969&       0.254\\
Countries           &          57&          82&          18&          42&          25\\
Observations        &        8101&        9003&         340&        1230&         956\\
R-square (ex. FE)   &        0.20&        0.23&        0.20&        0.17&        0.21\\

\midrule
\end{tabularx}
}
\end{center}
\vspace{-.5cm}\singlespacing {\footnotesize \textbf{Notes}: Conley standard errors, adjusted for spatial auto-correlation with a cutoff distance of 500km, are shown in parentheses. All regressions include province fixed effects, a constant, and controls for the district urbanization rate and log density of district nighttime lights. The coefficient estimate on rural population density indicates the value of $\beta_g$. Inclusion of districts is based on whether they have more than 50\% of their land area in the given K{\"o}ppen-Geiger zone. See text for details.
}
\end{table}

\clearpage
\begin{table}[!htb]
\begin{center}
\caption{Estimates of Land Elasticity, $\beta$, by Regions, 2000CE}
\label{TAB_beta_subregion}
{\footnotesize
\begin{tabularx}{\textwidth}{lXXXXX}
\midrule
\multicolumn{6}{l}{Dependent Variable in all panels: Log caloric yield ($A_{isg}$)} \\ \\
\multicolumn{6}{l}{Panel A} \\
 &          &         &             &  \multicolumn{2}{c}{Excl. China, Japan, Korea} \\ \cmidrule(lr){5-6}
 & North \& &         &              & South \&  & Central \&             \\
 & Western  & Eastern & Southern     & Southeast & West        \\
 & Europe   & Europe  & Europe       & Asia      & Asia      \\
 & (1) & (2) & (3) & (4) & (5) \\
\midrule
Log rural density ($\beta_g$)&       0.259&       0.287&       0.272&       0.152&       0.181\\
                    &     (0.036)&     (0.031)&     (0.041)&     (0.026)&     (0.024)\\
\midrule
p-value $\beta=0$   &       0.000&       0.000&       0.000&       0.000&       0.000\\
p-value $\beta=\beta_{NWEur}$&            &       0.539&       0.791&       0.017&       0.072\\
Countries           &          16&           9&           9&          13&          18\\
Observations        &        1684&        4821&        1137&        4312&        2878\\
R-square (ex. FE)   &        0.29&        0.34&        0.32&        0.21&        0.24\\

\midrule
\\
\multicolumn{6}{l}{Panel B} \\
 & Temperate & Tropical  & Tropical & South    & North    \\
 & Americas  & Americas  & Africa   & Africa   & Africa     \\
\midrule
Residuals           &       0.188&       0.113&       0.089&       0.134&       0.249\\
                    &     (0.030)&     (0.016)&     (0.014)&     (0.071)&     (0.014)\\
\midrule
p-value $\beta=0$   &       0.000&       0.000&       0.000&       0.059&       0.000\\
p-value $\beta=\beta_{NWEur}$&       0.133&       0.000&       0.000&       0.116&       0.827\\
Countries           &           5&          22&          39&           4&           5\\
Observations        &        3796&        9373&        3181&         198&        1220\\
R-square (ex. FE)   &        0.24&        0.12&        0.18&        0.27&        0.28\\

\midrule
\\
\multicolumn{6}{l}{Panel C} \\
 & All& Temperate & Sub-Tropical & & North \& \\
 & China & China  & China & Japan & South Korea  \\
 & (1) & (2) & (3) & (4) & (5) \\
\midrule
Log rural density   &       0.414&       0.518&       0.107&       0.155&       0.190\\
                    &     (0.083)&     (0.058)&     (0.026)&     (0.011)&     (0.061)\\
\midrule
p-value $\beta=0$   &       0.000&       0.000&       0.000&       0.000&       0.002\\
p-value $\beta=\beta^{NWEur}$&       0.102&       0.000&       0.001&       0.008&       0.309\\
Countries           &           1&           1&           1&           1&           2\\
Observations        &         266&         130&         136&        1039&         311\\
Adjusted R-square   &        0.25&        0.26&        0.21&        0.21&        0.21\\

\midrule

\end{tabularx}
}
\end{center}
\vspace{-.5cm}\singlespacing {\footnotesize \textbf{Notes}: Conley standard errors, adjusted for spatial auto-correlation with a cutoff distance of 500km, are shown in parentheses. All regressions include province fixed effects, a constant, and controls for the district urbanization rate and log density of district nighttime lights. See appendix for lists of exact countries included in each region. The coefficient estimate on rural population density indicates the value of $\beta_g$, see equation (\ref{EQ_regress}). The countries included in each region can be found in this appendix.
}
\end{table}


\clearpage
\begin{table}[!htb]
\begin{center}
\caption{Estimates of Land Elasticity, $\beta$, Expanded Definitions}
\label{TAB_beta_expand}
{\footnotesize
\begin{tabularx}{\textwidth}{lXXXXXX}
\midrule
\multicolumn{6}{l}{Dependent Variable: Log caloric yield ($A_{isg}$)} \\ \\
 &   & &                             & \multicolumn{3}{c}{Urban Pop. $<25K$} \\
 & \multicolumn{3}{c}{Suitable for:} & \multicolumn{3}{c}{Suitable for:} \\ \cmidrule(lr){2-4} \cmidrule(lr){5-7}
 & Temperate   & Any       & Any      & Temperate     & Any       & Any      \\
 & and Tropical & Temperate & Tropical & and Tropical & Temperate & Tropical \\
 & (1) & (2) & (3) & (4) & (5) & (6) \\
\midrule
Log rural density ($\beta_g$)&       0.140&       0.179&       0.132&       0.156&       0.202&       0.145\\
                    &     (0.013)&     (0.017)&     (0.011)&     (0.015)&     (0.020)&     (0.013)\\
\midrule
p-value $\beta=0$   &       0.000&       0.000&       0.000&       0.000&       0.000&       0.000\\
Countries           &         119&         137&         137&         110&         131&         130\\
Observations        &       15692&       26353&       24780&       11008&       18656&       17670\\
R-square (ex. FE)   &        0.14&        0.19&        0.13&        0.15&        0.21&        0.14\\

\midrule
\end{tabularx}
}
\end{center}
\vspace{-.5cm}\singlespacing {\footnotesize \textbf{Notes}: Conley standard errors, adjusted for spatial auto-correlation with a cutoff distance of 500km, are shown in parentheses. All regressions include province fixed effects, a constant, and controls for the district urbanization rate and log density of district nighttime lights. The coefficient estimate on rural population density indicates the value of $\beta_g$. ``Temperate and tropical'' includes all districts that are suitable for \textit{both} tropical and temperate crops (as defined in the text). ``Any temperate'' includes any district that is suitable for temperate crops (regardless of their suitability for tropical crops), and ``Any tropical'' is defined similarly for tropical crops regardless of suitability for temperate.
}
\end{table}


\end{document}