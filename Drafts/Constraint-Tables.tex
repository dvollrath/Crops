\documentclass[11pt]{article}
%%%%%%%%%%%%%%%%%%%%%%%%%%%%%%%%%%%%%%%%
\usepackage{amsmath}
\usepackage{verbatim}
\usepackage[usenames,dvipsnames]{color}
\usepackage{setspace}
\usepackage{lscape}
\usepackage{longtable}
\usepackage[top=1.25in,bottom=1.25in,left=1in,right=1in]{geometry}
\usepackage{graphicx}
\usepackage{epstopdf}
\usepackage[usenames,dvipsnames]{pstricks}
\usepackage{epsfig}
\usepackage{pstricks-add}
\usepackage{pst-node}
\usepackage{pst-plot}
\usepackage{fancyhdr}
\usepackage[absolute,showboxes]{textpos}
\usepackage{booktabs}
\usepackage{dcolumn}
\usepackage{arydshln}
\usepackage{natbib}
\usepackage{tabularx}
\usepackage{subfigure}

\newtheorem{proposition}{Proposition}
\newtheorem{corollary}{Corollary}

\setcounter{MaxMatrixCols}{10}
\newcolumntype{d}[1]{D{.}{.}{-2.#1}}
\newenvironment{proof}[1][Proof]{\noindent\textbf{#1.} }{\ \rule{0.5em}{0.5em}}
\setlength{\columnsep}{.2in}
\psset{unit=1cm}

\def\sym#1{\ifmmode^{#1}\else\(^{#1}\)\fi}

\begin{document}
\begin{titlepage}
\vspace{2in} \noindent {\large \today}

\vspace{.5in} \noindent {\Large \textbf{\strut How Tight are Malthusian Constraints?}}

\vspace{.25in} \noindent {\large T. Ryan Johnson}

\vspace{.05in} \noindent University of Houston

\vspace{.25in} \noindent {\large Dietrich Vollrath}

\vspace{.05in} \noindent University of Houston

\vfill \noindent \textsc{Abstract} \hrulefill

\vspace{.05in} \noindent We provide a methodology to estimate the elasticity of agricultural output with respect to land - the Malthusian constraint - using variation in rural densities across different locations. We use district-level data from around the globe on rural densities and inherent agricultural productivity to estimate the elasticity for various sub-samples. We find the elasticity is highest in areas suitable for temperate crops (e.g. wheat and rye), and lowest in areas suitable for sub-tropical crops (e.g. cassava and rice). We show theoretically that a higher elasticity results in greater sensitivity of non-agricultural employment and income per capita to shocks in population size and productivity, and confirm this with evidence from the post-war mortality transition.

\vspace{.1in} \hrule

\vspace{.5in} \noindent {\small JEL Codes: O1, O13, O44, Q10}

\vspace{.1in} \noindent {\small Keywords: land constraints, Malthusian stagnation, agriculture}

\vspace{.1in} \noindent {\small Contact information: 201C McElhinney Hall, U. of Houston, Houston, TX 77204, devollrath@uh.edu. We thank Francesco Caselli, Martin Fiszbein, Oded Galor, Nippe Lagerl{\"o}f, Debin Ma, Stelios Michalopolous, Nathan Nunn, {\"O}mer {\"O}zak, Enrico Spolaore, Joachim Voth, and David Weil, as well as seminar participants at the London School of Economics and the Brown Conference on Deep-rooted Determinants of Development for their comments. All errors remain our own.}
\end{titlepage}

\pagebreak

\clearpage

\begin{table}[!htb]
\begin{center}
\caption{Summary Statistics for District Level Data, 2000CE}
\label{TAB_summ}
{\small
\begin{tabularx}{\textwidth}{lXXXXXXX}
\midrule
 &      &            & \multicolumn{5}{c}{Percentiles:} \\ \cmidrule{4-8}
 & Mean & SD  & 10th    & 25th    & 50th & 75th & 90th \\
\midrule
Labor/land (persons/ha) &     0.76&     1.19&     0.05&     0.13&     0.34&     0.81&     1.92\\
Caloric yield (mil cals/ha) &    10.83&     4.83&     5.01&     7.20&    10.68&    13.83&    16.93\\
Log light density &    -2.97&     2.92&    -6.43&    -3.93&    -2.63&    -1.03&     0.20\\
Road density (km per sq-km) &     0.40&     0.53&     0.06&     0.11&     0.22&     0.47&     0.90\\
Share of roads, highway &     0.03&     0.09&     0.00&     0.00&     0.00&     0.00&     0.08\\
Share of roads, primary &     0.15&     0.20&     0.00&     0.00&     0.09&     0.22&     0.39\\
Share of roads, secondary &     0.33&     0.30&     0.00&     0.08&     0.24&     0.51&     0.82\\
Slope index &    70.78&    24.18&    33.64&    52.63&    77.69&    91.88&    97.05\\
Distance (km) to city w/ 100,000 &    61.20&    70.57&     6.89&    16.95&    40.33&    78.50&   137.32\\

\midrule
\end{tabularx}
}
\end{center}
\vspace{-.5cm}\singlespacing {\footnotesize \textbf{Notes}: A total of 32,862 observations for each variable (these come from 2,471 provinces in 154 countries). Caloric yield, $A_{isc}$ calculated by the authors using data from \citet{galorozak2016}. Rural density, $L_{Aisc}/X_{isc}$ calculated by the authors using data from \citet{hyde31} for rural population. Both caloric yield and rural density were trimmed at the 99th and 1st percentiles of their raw data prior to calculating the summary statistics in this table. Urbanization rate taken from \citet{hyde31}. Log mean light density derived from the Global Radiance Calibrated Nightime Lights data provided by NOAA/NGDC, as in \citet{hssw2016}. 
}
\end{table}

\clearpage

\begin{table}[!htb]
\begin{center}
\caption{Estimates of Malthusian Tightness, $\beta$, by Crop Suitability, 2000CE}
\label{TAB_beta_crops}
{\footnotesize
\begin{tabularx}{\textwidth}{lXXXXXX}
\midrule
\multicolumn{7}{l}{Dependent Variable in all panels: Log caloric yield ($A_{isc}$)} \\ \\
\multicolumn{7}{l}{Panel A: Samples defined by crop family (wheat vs. rice):} \\ \\
 & \multicolumn{2}{c}{By suitability:} & \multicolumn{2}{c}{By max calories:} & \multicolumn{2}{c}{By harvest area:}\\ \cmidrule(lr){2-3} \cmidrule(lr){4-5} \cmidrule(lr){6-7} 
 & Wheat & Rice & Wheat  & Rice  & Wheat  & Rice \\
 & Only & Only &  $>33\%$ & $>33\%$ & $>50\%$ & $>50\%$   \\
 & (1) & (2) & (3) & (4) & (5) & (6) \\
\midrule
Residuals           &       0.225&       0.130&       0.204&       0.134&       0.225&       0.104\\
                    &     (0.021)&     (0.018)&     (0.015)&     (0.012)&     (0.023)&     (0.015)\\
\midrule
p-value $\beta=0$   &       0.000&       0.000&       0.000&       0.000&       0.000&       0.000\\
p-value $\beta=\beta_{Temp}$&            &       0.001&            &       0.000&            &       0.000\\
Countries           &          91&          81&          74&          84&          88&          83\\
Observations        &       10661&        9088&       10708&        7564&       11618&       12292\\
Adjusted R-square   &        0.26&        0.21&        0.22&        0.19&        0.25&        0.20\\

\midrule
\\
\multicolumn{7}{l}{Panel B: Samples with other restrictions (using suitability to distinguish crop families)} \\ \\
 & \multicolumn{2}{c}{Urban Pop. $<25K$:} & \multicolumn{2}{c}{Ex. Europe/N. Amer.:} & \multicolumn{2}{c}{Rural dens. $>$ 25th P'tile:}\\ \cmidrule(lr){2-3} \cmidrule(lr){4-5} \cmidrule(lr){6-7}
  & Wheat Only& Rice Only & Wheat Only& Rice Only& Wheat Only& Rice Only\\
 & (1) & (2) & (3) & (4) & (5) & (6) \\
\midrule
Log labor/land ratio ($\beta_g$)&       0.300&       0.126&       0.296&       0.124&       0.298&       0.124\\
                    &     (0.045)&     (0.024)&     (0.048)&     (0.025)&     (0.045)&     (0.023)\\
\midrule
p-value $\beta_g=0$ &       0.000&       0.000&       0.000&       0.000&       0.000&       0.000\\
p-value $\beta_g=\beta_{Temp}$&            &       0.001&            &       0.001&            &       0.001\\
Countries           &          68&          66&          56&          36&          17&          62\\
Observations        &        7529&        6192&        6429&        4071&         813&        6676\\
R-square (ex. FE)   &        0.20&        0.16&        0.20&        0.17&        0.15&        0.11\\

\midrule
\end{tabularx}
}
\end{center}
\vspace{-.5cm}\singlespacing {\footnotesize \textbf{Notes}: Conley standard errors, adjusted for spatial auto-correlation with a cutoff distance of 500km, are shown in parentheses. All regressions include province fixed effects, a constant, and controls for the district urbanization rate and log density of district nighttime lights. The coefficient estimate on rural population density indicates the value of $\beta$, see equation (\ref{EQ_regress}). Rural population is from HYDE database \citep{hyde31}, and caloric yield is the author's calculations based on the data from \citet{galorozak2016}. Inclusion of districts in the regression is based on the listed criteria related to crop families. See text for all crops included in the wheat and rice families, and for details of the inclusion criteria.
}
\end{table}

\clearpage

\begin{table}[!htb]
\begin{center}
\caption{Estimates of Malthusian Tightness, $\beta$, by K{\"o}ppen-Geiger Zone, 2000CE}
\label{TAB_beta_kg}
{\footnotesize
\begin{tabularx}{\textwidth}{lXXXXXX}
\midrule
\multicolumn{7}{l}{Dependent Variable in all panels: Log caloric yield ($A_{isc}$)} \\ \\
\multicolumn{7}{l}{Panel A: Climate Zones} \\
 & Equatorial & Arid & Temperate & Snow  &     &   \\
 & (1) & (2) & (3) & (4) &  & \\
\midrule
Log rural density   &       0.120&       0.156&       0.172&       0.236\\
                    &     (0.016)&     (0.030)&     (0.020)&     (0.032)\\
\midrule
p-value $\beta=0$   &       0.000&       0.000&       0.000&       0.000\\
p-value $\beta=\beta^{Equa}$&            &       0.276&       0.033&       0.001\\
Countries           &          79&          55&          93&          40\\
Observations        &       10600&        2533&       12748&        5936\\
Adjusted R-square   &        0.11&        0.10&        0.15&        0.19\\

\midrule
\\
\multicolumn{7}{l}{Panel B: Precipitation Zones} \\
& Fully     & Dry         & Dry        &              &            & \\
& Humid & Summer & Winter & Monsoon & Desert & Steppe \\
 & (1) & (2) & (3) & (4) & (5) & (6) \\
\midrule
Log rural density   &       0.186&       0.184&       0.127&       0.139&       0.094&       0.115\\
                    &     (0.028)&     (0.027)&     (0.018)&     (0.023)&     (0.044)&     (0.027)\\
\midrule
p-value $\beta=0$   &       0.000&       0.000&       0.000&       0.000&       0.033&       0.000\\
p-value $\beta=\beta^{Fully}$&            &       0.947&       0.073&       0.190&       0.078&       0.072\\
Countries           &          97&          44&          74&          42&          29&          53\\
Observations        &       16216&        2978&        8503&        1655&         330&        2093\\
Adjusted R-square   &        0.19&        0.19&        0.17&        0.19&        0.19&        0.18\\

\midrule
\\
\multicolumn{7}{l}{Panel C: Temperature Zones} \\
    & Hot        & Warm        & Cool       & Hot      & Cold     &  \\
    & Summer & Summer & Summer & Arid & Arid &   \\
 & (1) & (2) & (3) & (4) & (5) &  \\    
\midrule
Log rural density ($\beta_g$)&       0.144&       0.219&       0.286&       0.179&       0.186\\
                    &     (0.014)&     (0.029)&     (0.054)&     (0.033)&     (0.034)\\
\midrule
p-value $\beta=0$   &       0.000&       0.000&       0.000&       0.000&       0.000\\
p-value $\beta=\beta_{Humid}$&            &       0.004&       0.010&       0.289&       0.248\\
Countries           &          61&          84&          26&          43&          27\\
Observations        &        9312&        9858&         540&        1582&        1160\\
R-square (ex. FE)   &        0.17&        0.23&        0.18&        0.16&        0.19\\

\midrule
\end{tabularx}
}
\end{center}
\vspace{-.5cm}\singlespacing {\footnotesize \textbf{Notes}: Conley standard errors, adjusted for spatial auto-correlation with a cutoff distance of 500km, are shown in parentheses. All regressions include province fixed effects, a constant, and controls for the district urbanization rate and log density of district nighttime lights. The coefficient estimate on rural population density indicates the value of $\beta$, see equation (\ref{EQ_regress}). Rural population is from HYDE database \citep{hyde31}, and caloric yield is the author's calculations based on the data from \citet{galorozak2016}. Inclusion of districts is based on whether they have more than 50\% of their land area in the given K{\"o}ppen-Geiger zone. See text for details.
}
\end{table}

\clearpage
\begin{table}[!htb]
\begin{center}
\caption{Estimates of Malthusian Tightness, $\beta$, by Regions, 2000CE}
\label{TAB_beta_subregion}
{\footnotesize
\begin{tabularx}{\textwidth}{lXXXXX}
\midrule
\multicolumn{6}{l}{Dependent Variable in all panels: Log caloric yield ($A_{isc}$)} \\ \\
\multicolumn{6}{l}{Panel A} \\
 &          &         &             &  \multicolumn{2}{c}{Excl. China, Japan, Korea} \\ \cmidrule(lr){5-6}
 & North \& &         &              & South \&  & Central \&             \\
 & Western  & Eastern & Southern     & Southeast & West        \\
 & Europe   & Europe  & Europe       & Asia      & Asia      \\
 & (1) & (2) & (3) & (4) & (5) \\
\midrule
Log rural density ($\beta_g$)&       0.159&       0.303&       0.278&       0.145&       0.177\\
                    &     (0.041)&     (0.078)&     (0.085)&     (0.039)&     (0.033)\\
\midrule
p-value $\beta=0$   &       0.000&       0.000&       0.001&       0.000&       0.000\\
p-value $\beta=\beta_{NWEur}$&            &       0.098&       0.201&       0.812&       0.730\\
Countries           &          16&           9&           8&          13&          18\\
Observations        &        1412&        4812&         947&        3184&        2208\\
R-square (ex. FE)   &        0.34&        0.33&        0.38&        0.26&        0.24\\

\midrule
\\
\multicolumn{6}{l}{Panel B} \\
 & Temperate & Tropical  & Tropical & South    & North    \\
 & Americas  & Americas  & Africa   & Africa   & Africa     \\
\midrule
Log rural density ($\beta_g$)&       0.137&       0.079&       0.071&       0.103&       0.363\\
                    &     (0.040)&     (0.015)&     (0.017)&     (0.071)&     (0.048)\\
\midrule
p-value $\beta=0$   &       0.001&       0.000&       0.000&       0.145&       0.000\\
p-value $\beta=\beta_{NWEur}$&       0.710&       0.071&       0.063&       0.502&       0.002\\
Countries           &           5&          22&          39&           4&           5\\
Observations        &        3011&        8653&        2638&         154&        1186\\
R-square (ex. FE)   &        0.24&        0.13&        0.22&        0.32&        0.32\\

\midrule
\\
\multicolumn{6}{l}{Panel C} \\
 & All& Temperate & Sub-Tropical & & North \& \\
 & China & China  & China & Japan & South Korea  \\
 & (1) & (2) & (3) & (4) & (5) \\
\midrule
Log rural density ($\beta_g$)&       0.259&       0.369&       0.054&       0.121&       0.139\\
                    &     (0.068)&     (0.059)&     (0.029)&     (0.015)&     (0.010)\\
\midrule
p-value $\beta=0$   &       0.000&       0.000&       0.066&       0.000&       0.000\\
p-value $\beta=\beta_{NWEur}$&       0.209&       0.004&       0.042&       0.402&       0.650\\
Countries           &           1&           1&           1&           1&           2\\
Observations        &         190&          99&          91&         848&         237\\
R-square (ex. FE)   &        0.35&        0.35&        0.34&        0.36&        0.32\\

\midrule

\end{tabularx}
}
\end{center}
\vspace{-.5cm}\singlespacing {\footnotesize \textbf{Notes}: Conley standard errors, adjusted for spatial auto-correlation with a cutoff distance of 500km, are shown in parentheses. All regressions include province fixed effects, a constant, and controls for the district urbanization rate and log density of district nighttime lights. See appendix for lists of exact countries included in each region. The coefficient estimate on rural population density indicates the value of $\beta$, see equation (\ref{EQ_regress}). Rural population is from HYDE database \citep{hyde31}, and caloric yield is the author's calculations based on the data from \citet{galorozak2016}. The countries included in each region can be found in the appendix.
}
\end{table}

\clearpage
\begin{table}[!htb]
\begin{center}
\caption{Panel Estimates of Effect of Population Change, by Tightness of Malthusian Constraint}
\label{TAB_pop_panel}
{\footnotesize
\begin{tabularx}{\textwidth}{lXXXXXX}
\midrule
 & \multicolumn{6}{c}{Dependent Variable:} \\ \cmidrule(lr){2-7}
 & \multicolumn{2}{c}{Log GDP per capita} & \multicolumn{2}{c}{Log GDP per worker} & \multicolumn{2}{c}{Log population} \\ \cmidrule(lr){2-3} \cmidrule(lr){4-5} \cmidrule(lr){6-7}
 & $\beta<$Median & $\beta>$Median & $\beta<$Median & $\beta>$Median & $\beta<$Median & $\beta>$Median \\
 & (1) & (2) & (3) & (4) & (5) & (6) \\
\midrule
 & \multicolumn{6}{c}{Panel A:} \\ \cmidrule(lr){2-7}
Mortality rate      &       0.333&       0.723&       0.284&       0.776&      -0.361&      -0.597\\
                    &     (0.271)&     (0.136)&     (0.262)&     (0.145)&     (0.186)&     (0.152)\\
\midrule
p-value $\theta=0$  &       0.220&       0.000&       0.281&       0.000&       0.054&       0.000\\
p-value $\theta=\theta^{Below}$&           .&       0.199&           .&       0.102&           .&       0.327\\
Countries           &          16&          16&          16&          16&          16&          16\\
Observations        &         128&         128&         128&         128&         128&         128\\

\midrule
\\
 & \multicolumn{6}{c}{Panel B:} \\ \cmidrule(lr){2-7}
Log life expectancy &      -0.204&      -1.665&      -0.196&      -1.721&       1.578&       1.869\\
                    &     (0.425)&     (0.259)&     (0.405)&     (0.267)&     (0.214)&     (0.249)\\
\midrule
p-value $\theta=0$  &       0.632&       0.000&       0.629&       0.000&       0.000&       0.000\\
p-value $\theta=\theta^{Loose}$&           .&       0.004&           .&       0.002&           .&       0.377\\
Countries           &          16&          17&          16&          17&          16&          17\\
Observations        &         120&         129&         120&         129&         120&         129\\

\midrule
\\
 & \multicolumn{6}{c}{Panel C:} \\ \cmidrule(lr){2-7}
Log population      &      -0.562&      -0.693&      -0.534&      -0.749\\
                    &     (0.092)&     (0.063)&     (0.089)&     (0.058)\\
\midrule
p-value $\theta=0$  &       0.000&       0.000&       0.000&       0.000\\
p-value $\theta=\theta^{Below}$&           .&       0.245&           .&       0.044\\
Countries           &          30&           5&          30&           5\\
Observations        &         238&          40&         238&          40\\

\midrule
\end{tabularx}
}
\end{center}
\vspace{-.5cm}\singlespacing {\footnotesize \textbf{Notes}: Robust standard errors are reported in parentheses. All regressions include both year fixed effects and country fixed effects. The value of $\beta$ for each country was found by estimating equation (\ref{EQ_regress}) separately for each, including province-level fixed effects. Countries are then included in a regression here based on how their $\beta$ compares to the median from the 34 countries. The mortality rate used as an explanatory variable in Panel A is the mortality rate from 15 infectious diseases, as documented by \cite{aj07}. All data on GDP per capita, GDP per worker, population, and life expectancy is also taken directly from those authors dataset. The p-value of $\theta = \theta^{Loose}$ is from a test that the estimated coefficient in a column (with $\beta$ over the median) is equal to the coefficient in the column immediately preceding it (with $\beta$ under the median).
}
\end{table}



\end{document}